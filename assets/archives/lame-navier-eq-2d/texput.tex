\documentclass{article}

\usepackage{amsmath}
\usepackage{amssymb}
\usepackage{amsthm}
\usepackage[margin = 2.54cm]{geometry}
\usepackage{tikz}

\newtheorem{lemma}{Lemma}
\newtheorem{definition}[lemma]{Definition}
\newtheorem{prop}[lemma]{Proposition}

\begin{document}
We start from the following constitutive law.
\begin{prop}[The constitutive law of three-dimensional isotropic meterial]
  \begin{equation}
    \sigma^{ij} = \lambda \epsilon^{kk} \delta^{ij} + 2G \epsilon^{ij},
  \end{equation}
  where $\lambda$ and~$G$ is defined as
  \begin{equation}\begin{aligned}
    \lambda &\mathrel{:=} \frac{\nu E}{(1 + \nu)(1 - 2\nu)}, \\
    G &\mathrel{:=} \frac{E}{2(1 + \nu)}.
  \end{aligned}\end{equation}
\end{prop}
See Theory of Elasticity, Timoshenko, for reference.

The governing equation.
\begin{prop}[The governing equation of material in statics]
  \begin{equation}
    \frac{\partial \sigma^{ij}}{\partial x^j} + \rho \! f^i = 0.
  \end{equation}
\end{prop}

And the kinematic equation.
\begin{definition}
  [The kinematic description of material with infinitesimal deformation]
  \begin{equation}
    \epsilon^{ij} \mathrel{:=} \frac{1}{2} \biggl(
    \frac{\partial u^i}{\partial x^j} + 
    \frac{\partial u^j}{\partial x^i} \biggr).
  \end{equation}
\end{definition}

A substitution gives
\begin{equation}
  (\lambda + G) \frac{\partial}{\partial x^i}
  \biggl( \frac{\partial u^j}{\partial x^j} \biggr) +
  G \frac{\partial u^i}{\partial x^j \partial x^j} +
  \rho \! f^i = 0,
\end{equation}
which is the three-dimensional Lam\'e-Naiver equation, in statics.

Now consider a plane stress problem, where $\sigma^{i3} = 0$,
and we have
\begin{equation}
  \rho \! f^3 = 0.
\end{equation}
Meanwhile, we have
\[
  \sigma^{33} =
  \lambda \frac{\partial u^j}{\partial x^j} + 
  2 G \frac{\partial u^3}{\partial x^3} = 0, 
\]
which gives
\[
  \frac{\partial u^3}{\partial x^3} =
  -\frac{\nu}{1 - 2\nu} \frac{\partial u^j}{\partial x^j}
  \qquad \text{and} \qquad
  \frac{\partial u^\beta}{\partial x^\beta} =
  \frac{1 - \nu}{1 - 2\nu} \frac{\partial u^j}{\partial x^j}.
\]
Also, we have
\[
  \sigma^{\alpha 3} =
  G \biggl( \frac{\partial u^\alpha}{\partial x^3} +
  \frac{\partial u^3}{\partial x^\alpha} \biggr) = 0,
\]
which gives
\[
  \frac{\partial u^\alpha}{\partial x^3 \partial x^3} =
  - \frac{\partial}{\partial x^\alpha} \frac{\partial u^3}{\partial x^3}.
\]
Hence, for a plane stress problem, we have
\[
  \biggl( \frac{1 - 2\nu}{1 - \nu} (\lambda + G) -
  \frac{\nu}{1 - \nu} G \biggr)
  \frac{\partial}{\partial x^\alpha}
  \frac{\partial u^\beta}{\partial x^\beta} +
  G \frac{\partial^2 x^\alpha}{\partial x^\beta x^\beta} +
  \rho \! f^\alpha = 0,
\]
or
\begin{equation}
  \frac{E}{2(1 - \nu)} \frac{\partial}{\partial x^\alpha}
  \frac{\partial u^\beta}{\partial x^\beta} +
  G \frac{\partial^2 x^\alpha}{\partial x^\beta x^\beta} +
  \rho \! f^\alpha = 0.
\end{equation}

By using the transform from a plane stress problem to a plane strain one,
we have
\begin{equation}
  \frac{E}{2(1 + \nu)(1 - 2\nu)} \frac{\partial}{\partial x^\alpha}
  \frac{\partial u^\beta}{\partial x^\beta} +
  G \frac{\partial^2 x^\alpha}{\partial x^\beta x^\beta} +
  \rho \! f^\alpha = 0.
\end{equation}
\end{document}