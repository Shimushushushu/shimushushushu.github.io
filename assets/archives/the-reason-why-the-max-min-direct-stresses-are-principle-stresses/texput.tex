\documentclass{article}

\usepackage{amsmath}
\usepackage{geometry}

\geometry{
  top = 2.54cm, bottom = 2.54cm,
  left = 3.18cm, right = 3.18cm
}

\begin{document}
Since the plain stress matrix is symmetric, there's a set of eigenvectors, $\bar{v}_{1}$ and $\bar{v}_{2}$, which form a orthonormal basis. So for all vector $\bar{n}$, $\bar{n}^{\mathrm{T}} \underline{\underline{\sigma}} \bar{n}$ can be rewritten as
\[
\sigma_{n} = \bar{n}^{\mathrm{T}} \underline{\underline{\sigma}} \bar{n} = 
(k_{1} \bar{n} + k_{2} \bar{n})^{\mathrm{T}} \underline{\underline{\sigma}} (k_{1} \bar{n} + k_{2} \bar{n}) = k_{1}^2 \lambda_{1} + k_{2}^2 \lambda_{2} \in [\lambda_{2}, \lambda_{1}],
\]
where $\lambda_{1}$ and $\lambda_{2}$ are eigenvalues of $\bar{v}_{1}$ and $\bar{v}_{2}$ respectively, and $\lambda_{1} \ge \lambda_{2}$. And $\sigma_{n}$ reaches its maximum or minimum, if and only if $\bar{n}$ is in the direction of $\bar{v}_{1}$ or $\bar{v}_{2}$, respectively, which is equivalent to the vanish of shear stresses.
\end{document}