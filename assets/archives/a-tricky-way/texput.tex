\documentclass{article}

\usepackage{amsmath}
\usepackage{geometry}
\usepackage{siunitx}

\geometry{
  left=3.18cm, right=3.18cm,
  top=2.54cm, bottom=2.54cm
}

\begin{document}
Suppose we're solving such a problem, where the beam is uniform (that is, $H_{33}^{\text{c}}$ remains unchanged along the whole beam), and it is subjected to both distributed but uniform transverse load and concentrated loads, in the $x_2$ direction. Then we have the following propositions:
\begin{enumerate}
  \item $\bar{u}_2$ and $\phi_3$ are continuous along the whole beam, by its geometry.
  \item $M_3$ is continuous along the whole beam, by the moment equilibrium equation.
  \item For every place, $x_0$ for example, $-V_2(x_0^-) + V_2(x_0^+) + F(x_0) = 0$, by the force equilibrium equation, where $F(x_0)$ is the concentrated force applied on that place. And we can rewrite the equation as
  \begin{equation}
  V_2 \vert_{x_0^-}^{x_0^+} + F(x_0) = 0.
  \label{eq:V_2}
  \end{equation}
  \item $\dfrac{\mathrm{d} V_2}{\mathrm{d} x_1} + p_2 = 0$ holds along the whole beam, or at least for places where no concentrated forces are applied.
\end{enumerate}
Of course there are some other boundary conditions, but we'll talk about that later.

Then, with those propositions, and assumptions we have made, we can construct a function $\hat{p}(x_1)$, which is
\begin{equation}
\hat{p}(x_1) = \dfrac{p_2}{H_{33}^{\text{c}}} + \sum_i \dfrac{F_i}{H_{33}^{\text{c}}} \delta(x_1 - x_i).
\label{eq:phat}
\end{equation}
Note that this function is valid in dimension, since the dimension of the Dirac $\delta$ function is $\SI{}{/m}$\footnote{Usually we use that function in signals, where the variable is usually $t$, and the dimension of $\delta(t - t_0)$ is $\SI{}{s^{-1}}$.}.

We integrate it by once and get
\[
\dfrac{p_2}{H_{33}^{\text{c}}} x_1 + \sum_i \dfrac{F_i}{H_{33}^{\text{c}}} u(x_1 - x_i) + C_1.
\]
We multiply it by $-H_{33}^{\text{c}}$, and call it $\hat{V}_2$, which is
\begin{equation}
\hat{V}_2(x_1) = -p_2 x_1 - \sum_i F_i u(x_1 - x_i) - H_{33}^{\text{c}} C_1,
\label{eq:vhat}
\end{equation}
and we surprisingly find that $\hat{V}_2$ satisfies equation~\ref{eq:V_2}. So we integrate equation~\ref{eq:vhat} and add a minus sign to obtain so-called $\hat{M}_3$, which is
\begin{equation}
\hat{M}_3(x_1) = \dfrac{p_2}{2} x_1^2 + \sum_i F_i (x_1 - x_i) u(x_1 - x_i) + H_{33}^{\text{c}} C_1 x_1 + C_2,
\label{eq:mhat}
\end{equation}
which is indeed continuous. In a similar fashion, we divide equation~\ref{eq:mhat} by $M_{33}^{\text{c}}$ and integrate it by once and twice respectively, we can obtain
\begin{align}
\hat{\phi}_3(x_1) &= \dfrac{p_2}{6 H_{33}^{\text{c}}} x_1^3 + \sum_i \dfrac{F_i}{2 H_{33}^{\text{c}}} (x_1 - x_i)^2 u(x_1 - x_i) + \dfrac{C_1}{2} x_1^2 + C_2 x_1 + C_3, \label{eq:phihat} \\
\hat{\bar{u}}_2(x_1) &= \dfrac{p_2}{24 H_{33}^{\text{c}}} x_1^4 + \sum_i \dfrac{F_i}{6 H_{33}^{\text{c}}} (x_1 - x_i)^3 u(x_1 - x_i) + \dfrac{C_1}{6} x_1^3 + \dfrac{C_2}{2} x_1^2 + C_3 x_1 + C_4, \label{eq:uhat}
\end{align}
where there are still four constants to be determined. So we'll use the boundary conditions. For example, for a simply supported beam, the displacement $\bar{u}_2$ and bending moment $M_3$ are $0$, and we can use equation \ref{eq:uhat}~and~\ref{eq:mhat} respectively to obtain those constants. What's more, we explicitly give equation~\ref{eq:phat} instead of some function that satisfies condition 4, because equation~\ref{eq:uhat} can be obtained by simply integrate equation~\ref{eq:phat} by four times.
\end{document}